\documentclass[sigconf]{acmart}

\title{Self-Supervised Learning for ECG Signal Analysis}

\author{Your Name}
\affiliation{%
  \institution{Your Institution}
  \city{Your City}
  \country{Your Country}
}
\email{your.email@example.com}

\author{Co-Author Name}
\affiliation{%
  \institution{Co-Author Institution}
  \city{Co-Author City}
  \country{Co-Author Country}
}
\email{coauthor.email@example.com}

\begin{document}

\begin{abstract}
  This paper presents a self-supervised learning approach for analyzing electrocardiogram (ECG) signals. We explore the potential of self-supervised techniques to improve the performance of ECG classification tasks without the need for extensive labeled datasets. Our experiments demonstrate that self-supervised learning can effectively capture the underlying patterns in ECG data, leading to improved classification accuracy.
\end{abstract}

\keywords{Self-Supervised Learning, ECG Signal Analysis, Machine Learning, Health Informatics}

\maketitle

\section{Introduction}
Electrocardiogram (ECG) signals are crucial for diagnosing various cardiovascular diseases. Traditional machine learning approaches often require large amounts of labeled data, which can be challenging to obtain in medical domains. Self-supervised learning (SSL) offers a promising alternative by leveraging unlabeled data to learn useful representations.

\section{Related Work}
Recent advancements in self-supervised learning have shown significant improvements in various domains, including image and speech processing. In the context of ECG analysis, several studies have explored the use of deep learning techniques, but few have focused on self-supervised methods.

\section{Methodology}
Our proposed method consists of the following steps:
\begin{itemize}
    \item Data Preprocessing: We preprocess the ECG signals to remove noise and artifacts.
    \item Self-Supervised Learning Framework: We design a framework that utilizes contrastive learning to learn representations from unlabeled ECG data.
    \item Classification: The learned representations are fine-tuned on a labeled dataset for classification tasks.
\end{itemize}

\section{Experiments}
We evaluate our method on publicly available ECG datasets. The results indicate that our self-supervised approach outperforms traditional supervised methods, achieving higher accuracy and robustness.

\begin{table}
\caption{Experimental Results}
\begin{tabular}{ccc}
\hline
Method & F1(probe) & F1(scratch) \\
\hline
FocusECG(ours) & 85.2 & 86.1 \\

\hline
\end{tabular}
\end{table}






\section{Conclusion}
This study demonstrates the effectiveness of self-supervised learning for ECG signal analysis. Our findings suggest that SSL can significantly enhance the performance of machine learning models in medical applications, paving the way for future research in this area.

\begin{acks}
We would like to thank the contributors of the ECG datasets used in this study. This research was supported by [Your Funding Source].
\end{acks}

\begin{thebibliography}{99}
\bibitem{ref1} Author, A. (Year). Title of the paper. \textit{Journal Name}, Volume(Issue), Page numbers.
\bibitem{ref2} Author, B. (Year). Title of the paper. \textit{Journal Name}, Volume(Issue), Page numbers.
\end{thebibliography}

\end{document}